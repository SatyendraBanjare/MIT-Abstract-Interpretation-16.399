\chapter{Abstract Invariance and Termination Proofs}

\begin{itemize}

	\item{\textbf{Fundamental Limitation} : The Hilbert's Axiomatic program to reform the basis of mathematics and it being proved incorrect by Godel's Incompleteness theorem. The Axiomatic system is not possible entirely !! 
	}

	\item{\textbf{Decidable/Semidecidable/Undecidable} : Decidable if algorithm exists to solve the program in finite time.
	Undecidable if there exists no algorithm to solve the problem in finite time.
	Semidecidable if solvable in finite time for some cases and completely un-solvable for other cases. 
	}

	\item{\textbf{Termination Problem} : Given input data and some preconditions, whether that sequential program will terminate or not. Non-Termination problem to check whether the program will not terminate.

	Using a simple example of Interpretor program, it has been shown that termination can be semi-decidable as well as Undecidable using 'reductio ad absurdum'.

	Non-Termination is not semidecidable. Simply anding semidecidable(termination) and semidecidable(Non-Termination) it results to decidable(termination).
	}

	\item{\textbf{Problem Reduction} : it follows that if program 'P' is decidable, it means its termination proof is also 'decidable'.

	Constant propogation problem (retaining constant value of a variable in a program) is undecidable !!
	}

	\item{\textbf{Approximations} : As a way to tackle (thought it is absurd to do completely) , We can have approximations of things that might happen. example - false alarms in ASTREE.
	}

	\item{\textbf{Computational Complexity} : 
	Time and space required to solve a problem. It has classes such as P , NP, Co-NP

	\begin{itemize}
		\item{\textbf{NP-Hard} is observed if every NP problem can be redued to it. }
		\item{\textbf{Boolean SAT problem} They are NP-complete Problems.

		NP-complete problems are generally solved by a fixed parameter approach.

		Probabilistic nature of solution algorithms has overall good average range of solving time.

		Approximations with optimal error range is done to solve the problem.

		Example of SAT solver algo - DLL algorithm.	
		}
	\end{itemize}
	}

	\item{\textbf{ Termination Analysis Method} : 
	It proves the termination conditions of  a program. Example of a while loop is explained. This methos uses 4 major parameters in consideration : Termination precondition , Loop invariant, operational semantics and Ranking Function.

	Forward reachability properties and partial correctness is referred to guide the program beginning, Backward reachability for the program termination. 
	

	\begin{itemize}
	\item{\textbf{Auxiliary termination counter method} : It involves using an auxilary counter to enforce the bound (termination) of program. The Loop invariant is found by using a forward relational static analysis. The operational semantics is then known by performing a forward static analysis.

	\textbf{Floyd's Ranking Method} to prove Termination then uses a ranking function to determine rank of loop/program invariant. This Ranking function should be always decreasing and Non-Negative.

	Example of arithmetic mean function's termination using ranking function is beautifully explained. 
	}
	
	\item{One method involves Expressing the loop invariant and relational semantics as numerical positivity constraints. Then the floyd's method is applied to get the termination conditions. 

	The Loop Invariant for a given step of loop can be expressed as an anding of variables of op-semnatics at each previous step !!. see equation in the slides.
	}

	\item{\textbf{Lagrangian Relaxation} : Involves reforming the problem by estimation the conjuctiona and implications using lagrangian relaxation.

	It is simple mathematically to understand $\lambda _ i$ lagrangian coefficients for the given problem. Finding these coefficients is carried using different methods. Basically we are approximating a major problem/ major curve by a smaller/easier representation.

	Yakubovich’s procedure is often used to prove the completeness.
	}

	\item{\textbf{Parametric Abstraction of Ranking Function} :
	Assuming a Ranking Function form for some unkown parameters we are trying to solve for those parameters.
	}

	\item{\textbf{Approximating Universal Quantifications using Linear Matrix Inequalities (LMI)} 
	LMI of ranking function is formed. Further problem reduction shows that this LMI problem reduces to provinf semi-definitiveness of matrix which can be done using mathematical programming and/ord using LMI solvers.

	Extending this, we should try to solve the Convex constraints (see slides pictures) also by the semidefinite programming. 

	This involves using linear programming, ellipsoidlal methods and interior methods.

	Futher more we can transform non-convex constraints to convex constrains then aplly the previous steps to solve things out.
	}
	
	\item{\textbf{Handling Non-determinancy} includes case analysis, interleaving for concurrency and encoding with explicit scheduler.}

	\item{\textbf{Floyd's method for invariance } involves determining loop precondition and loop invariant, further also proving that the loop variant is inductive.

	This transforms the problem to a Bilinear Matrix Inverse type problem.

	\textbf{BMI solvers by Semi-Definite programming} is thus used. 
	}

	\end{itemize}
	}

\end{itemize}

