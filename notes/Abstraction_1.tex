\chapter{Abstraction 1}


\begin{itemize}
	\item{\textbf{Informal introduction to abstract interpretation} : using the graphical language, the abstract interpretation is introduced. using objects to represent an abstract idea.
	}

	\item{\textbf{Concretization} : mathematically the lfp satisfies the constrants, our abstract objects also satisfy the constraints. thus abstract objects (interpretations) also seem to behave likewise.

	iteration and upper approximation form the set of satisfying points . 
	}

	\item{\textbf{abstraction examples} the abstraction function maps elements to their abstract definitions. 

	the Concretization examples include the galois connections 
	}

	\item{\textbf{Abstract ordering} : for the elements of set satisfying a given ordering, the abstraction function defined over them also follows the same ordering.
	}

	\item{\textbf{Abstract Rotations} : mathematically defined method to prove graphical rotations on the abstract elements.
	}

	\item{\textbf{Abstract transformer} : mathematically it defines the lfp called as abstract fixpoint for given abstract definition. graphically it means the abstraction defined on the petals (abstract objects).
	}

	\item{\textbf{Property Abstractions} :Properties are a set of objects (or set of states) satisfying some given conditions.

	it can be understood as variables, statements, heaps etc.

	Property abstraction deals with defining those sets mathematically  which be represented as lattices.
	}

	\item{\textbf{Abstraction, informal introduction} : Abstraction replace someting “concrete” with a schematic description that account for some, and in general not all properties, either known or inferred i.e.  an “abstract” model or concept.

	it cannot define all the properties as concrete model.
	}

	\item{\textbf{Direction of abstraction} : can be abstraction from above or below in reference to the set of elements. Other schema may be using probabilistic reasoning to define the direction of abstraction.

	the from-above/below abstraction can be thought of abstracting the points of the x-y graph from up tp below or below to up.   
	}

	\item{\textbf{Minimal abstraction} : in the absence of an upper abstractions, finite approximations are used.

	the minimal abstraction is defined for the smallest defined property for a given set/lattice.

	minimal upper-approximations are carried out if abstractions are not definable there.

	A classical example of absence of minimal abstract upper-approximations is that of a disk with no mini-mal convex polyhedral approximation.

	there may be many minimal abstraction possible for the same property however the most cost-efficient abstraction should be used.

	example : rule of signs
	}

	\item{\textbf{Best abstraction} :  is an abstraction of property 'A' for the set 'P', is such the glb of P lies inside A.
	}

	\item{\textbf{The abstract domain is a Moore Family.}}

	\item{\textbf{closure operator based abstraction} : defines mapping from the lattice set P to abstract property set A.

	for Best abstractions, the Moore family definitions and the closure operator based definitions are equivalent.
	}

	\item{\textbf{Generalizing to complete lattices} : reasonings/mappings on abstraction property defined on a lattice can be generalized into complete lattices.

	this helps in forming a compositional approach to define all the complete abstraction property.
	}

	\item{\textbf{Specification of an abstract domain by a Galois surjection} : Correspondance between the concrete and abstract properties can be explained using compositional rules of a surjective operator and its inverse. this compostion con further be explained using galois connections using an abstraction function and concretization operator.
	}

	\item{\textbf{Abstract domains are complete lattices in the presence of best abstractions} : as stated previously by the use of closure operators and moore family analogy.
	}

	\item{\textbf{Standard examples of abstractions formalized by Galois connections} : 

	\begin{itemize}

		\item{\textbf{Subset restriction abstraction}}

		\item{\textbf{Elementwise/homomorphic abstraction} :
		}

		\item{\textbf{Subset inclusion abstraction} :
		}

		\item{\textbf{Functional Abstraction} :
		}

		\item{\textbf{Relational Abstraction} :
		}

		\item{\textbf{Relational lattice Abstraction} :
		}

		\item{\textbf{minimal abstraction} :
		}

		\item{\textbf{interval abstraction} :
		}

		\item{\textbf{abstraction of function at a point} :
		}

		\item{\textbf{Abstraction of a function at a set of point} :
		}

		\item{\textbf{Abstraction of a set of functions by a function} :
		}

		\item{\textbf{Abstraction of lattice functionals} :
		}

		\item{\textbf{Cartesian abstraction of a setof pairs} :
		}

		\item{\textbf{Pointwise abstraction composition} :
		}

		\item{\textbf{Componentwise abstraction composition} :
		}

		\item{\textbf{Higher-order abstraction composition} :
		}

	\end{itemize}
	}

\end{itemize}