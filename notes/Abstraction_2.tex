\chapter{Abstraction 2}


\begin{itemize}
	\item{\textbf{Exact fixpoint Abstraction} : property transformer abstraction deals with finding out the abstraction operator value without having to find the lfp value for the whole lattice.

	this value ( or fixpoint, since we know they are equivalent) though can not be easily achieved, the method of higher order abstraction is used in accomplishing our goal.

	thus there are possible exact as well as approximate fixpoint abstraction.
	}

	\item{\textbf{Kleene exact fixpoint abstraction with Galois connections} : this is also called as fixpoint fusion, fixpoint lifting , fixpoint morphisms etc.

	Given two lattices and an operaor defined mapping between them, the overall abstraction through galois connections can be broken to smaller galois connection. 

	this involves having a strictly increasing sets having a limit defined by the max of lattice's elements.

	it also inolves that the operator must be monotonous and the lfp exists for the lattice.
	}

	\item{\textbf{Kleene exact fixpoint abstraction with continuous abstraction} : using more abstraction operators, defined and composed over the set of operators eventually results in the lfp .
	}

	\item{\textbf{Tarski exact fixpoint abstraction} : it also defines the composition of operator to result in the same fixpoint value.
	}

	\item{\textbf{transition systems} : 
	has a set of states and a transition relation. can be reflexive transitive closure.
	}

	\item{\textbf{post images, post image galois connections} : post image galois connections involves establishing galois connection between the post images anf the abstraction domain.
	}

	\item{\textbf{reachable states in fixpoint iterations} : the states are reachable in the fixpoint iterations if the postimage of transtion relation is satisfied by the invariant.
	}

\end{itemize}