\chapter{First Order Logic}

\begin{itemize}
	\item{\textbf{Formal Logics} : It has a formal language, a model-theoretic semantics and a Deduction system. It needs to be sound and complete.
	}

	\item{\textbf{Propositional Logics} : Formula representation of meta theories often involving the boolean algebric terms.
	}

	\item{\textbf{Model-Theoretic classical propositional logic} : it adds the semantics to model behaviour defined by formulas. Can be understood as imporving the contextual interpretation of same code.
	}
	
	\item{\textbf{Models} : A modelling semantics associates with the formulaes and variables. A model for a formula exits if its modelling	 semantics exits.

	Formulaes are further classified into a satisfiable, unsatisfiable and valid.
	}
	
	\item{\textbf{Hilbert Deductive System} : This logic system uses some basic axioms along with an inference rule. The objects on which these axioms are applied are formulae / logical statements.

	The Hypotheses in analyzed based on deduction from base cases.
	}
	
	\item{\textbf{Proof} : Provability and proof of a statement $\phi$ from $\tau$ exists if  $\tau \in P(F)$.

	simple example cases have been explained. if you are familiar with proof assistants, this is rather the basic exmaple you would have solved.
	}
	
	\item{\textbf{Soundness and Consistency of Deductive System} : proof system is sound if provable formaulaes do hold. It is consistent if  there are no contradictory proofs. 
	}
	
	\item{\textbf{Normalization} : Negative form basically defines the negation of any formula. Normalization means expression the combination of formula in a better "conjuctive" (or-ed) form. It is just a method of representation but is very useful. example : CNF and negative CNF.
	}
	
	\item{\textbf{Syntax specification of first order logic system} : the important features of first order logic system includes 
		\begin{itemize}
			\item {Lexems, includes symbols, constants and variable }
			\item{Terms includes the objects whose value if determined and it can be constant, variable or a function.}
			\item{Atomic Formulae are used to represent the basic/initial facts/featueres of the proof system }
			\item{Free Variables}
			\item{First order syntax}
			\item{bound variables}
			\item{theories}
		\end{itemize}

	\textbf{Substitution} refers to replacement of variables by 'Terms' of logic system.
	}
	
	\item{\textbf{Semantics of first order logic system} : Includes Assigning Environment preconditions to the variables, Interpretation using a domain of Interpretation, Assignment (or Substition speaking syntactically ).  

	}
	
	\item{\textbf{Deduction System for first order logic} : using the basic two inference rules (modus peanus and generalization), we define the working of the deuction system.

	Two basic examples are explained.
	}

	\item{\textbf{Extending first order logic system} : involves incorporating the theorems as terms in a logic system. this thus extends the logic systems's generalization rules.
	}

	\item{\textbf{Properties of Hilbert Deduction System} : 
	It is sound, complete proof of which can be explained using leibnitz inequality rules. It is however Un-decidable.
	}
		
\end{itemize}
