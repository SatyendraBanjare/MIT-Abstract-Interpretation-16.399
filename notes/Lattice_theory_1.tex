\chapter{Lattice Theory 1}


\begin{itemize}

	\item{\textbf{Covering relations} for posets basically mean if one set contains elements of other. it covers the other set.}
	\item{\textbf{Inverse of Partial Order is a partial order.} :
	}

	\item{\textbf{Hasse Diagram} : line diagram of posets' points. it is shown that it cna be extended.
	}

	\item{\textbf{Antichain} : is a poset if two random elements can be equal.
	}

	\item{\textbf{chain} : is a poset if two random elements are either less than one another, a<b ob b<a both possible.
	}

	\item{\textbf{Chain Condition} : Ascending and Descending chain conditions, ACC \& DCC. Ascending if gradually the number of equal elments decrese on a less than or equal poset. descending if numbers increase.
	}

	\item{\textbf{Tree, Max - Min} : A poset tree means that it's down set is a woset.

	Max-min is same as top-bottom of a poset. 
	}

	\item{\textbf{A poset is non-infinite if it satisfies bith ACC and DCC conditions.}}

	\item{\textbf{Duality} : Dual of a poset is its inverse. Its hasse diagram is simple vertical reflection of original poset.

	duality principle says if a statement is true for a poset, it is also true for its dual.
	}

	\item{\textbf{ upset \& downset} :
	upset if the superset poset has all members greater than current poset's members. downset if lower.
	}

	\item{\textbf{ Directed Sets} :
	It means x less than or equal z \& y less than or equal z for the elements of a subset of a poset with property 'less than or equal' .
	}

	\item{\textbf{Upper and Lower bounds} : max min for a subset of a poset.
	}

	\item{\textbf{properties of lub/glb} : 
	lub/glb of a poset is unique.

	lub exists if and only if there is a bottom in poset and glb if there is a supremum in the poset.
	}

	\item{\textbf{Subset ordering} : functional values of elements of subset of a poset also follow the characteristic property of that poset.
	}

	\item{\textbf{Lattices} : are posets.
	join semi lattice if any two elements have a LUB.
	meet semi lattice if any two elements have a GLB.

	A lattice is both a join and a meet lattice.
	}

	\item{\textbf{Semi-Lattice properties} : they follow associativity, commutativity and idempotency.
	They also follow Duality principle.
	}

	\item{\textbf{SubLattices} : of a lattice is a lattice following the characteristic ordering property. all it's elements are subset of original lattice.
	}

	\item{\textbf{CPO \& Complete Lattices} : complete partial order (CPO) is a poset that its increasing chain of P has lub in P. 

	complete lattice is poset if its every subset has a lub in P.

	complete lattice is never empty and has both LUB and GLB.
	}

	\item{\textbf{Bottom and top of latttice} : max-min of a lattice . easy to refer from the hasse diagram.
	}

	\item{\textbf{Finite Lattices} : are non-infinite lattices and are complete lattices.
	}

	\item{\textbf{Boolean algebra for Lattice} :  
	They follows distributive , commutative , associativity and transitivity properties.
	}

	\item{\textbf{Distributive Lattice} : if elements follow distribuitve law with GLB and ULB as additon and multiplication operations.
	}

	\item{\textbf{Semi-infinite distributive lattice} : if they follow either infinite join distributivity or infinite meet distributivity.

	infinite if it satisfies both.
	}

	\item{\textbf{Complements} : of an element of a poset exists if their ULB is poset's bottom and their GLB is poset's top. They are not unique
	}

	\item{\textbf{bounded posets and lattice} : have both a maximum and a minimum.
	}

	\item{\textbf{Complemented lattice} : if every lement has a complement in itself.
	}

	\item{\textbf{Boolean Lattices} : A Boolean latticeis a complemented distributive lattice.
	}

\end{itemize}