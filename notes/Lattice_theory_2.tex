\chapter{Lattice Theory 2}


\begin{itemize}
	\item{\textbf{Moore Family} : is a subset of a poset such that glb of P is an element of M.

	it is a complete lattice
	}

	\item{\textbf{Moore Closure} : exists if the glb of subset of a moore family is an element of M.
	example - convex subset of a poset.

	}

	\item{\textbf{Linear sum of Posets} : is a poset, possible if glb of one is less than lub of other.
	}

	\item{\textbf{Combination of Posets} : bottom lifting adds a bottom to P. top lifting adds a top to P.
	}

	\item{\textbf{Flat ordering} is just simple ordering the equal elements of a poset, that lie on a some horizontal line in the hasse diagram.
	}

	\item{\textbf{Smashed \& Disjoint sum of posets} : 
	smashed means joining two posets with one same element that is glb of one poset and is also a lub of another. Is exactly one common point

	disjoint if there is no only one such common / same element. rather there can be two joining together the hasse diagrams to form donut like shape.
	}

	\item{\textbf{cartesian product of posets} : every possible cartesian set that can be made from poset's elements.
	}

	\item{\textbf{smashed cartesian product} : has only one top/bottom element. }

	\item{\textbf{cardinal power} of set x maps it to a poset with pointwise ordering. }

\end{itemize}