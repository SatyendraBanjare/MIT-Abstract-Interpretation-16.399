\chapter{Non-relational monotonic finitary static analysis 2}

Abexp = abstract expression.

Bexp = boolean expression.

comm = command.


\begin{itemize}
	\item{\textbf{Generic backward/bottom-up static analysis of arithmetic expressions} : is not useful in tests where the end result is known.

	Is more helpful in the case of initialization and sign allotment.

	It follows that any information on the possible result of an arithmetic expression should bring informationon the values of the variables involved in the arith-metic expressions for its result to satisfy the knwon information.
	}

	\item{\textbf{Backward/bottom-up collecting semantics of arithmetic expressions is a lower closure operator} : collecting semantics of arithmetic helps in finding the environment variable for which the arithmentic expression will evaluate to true.

	A lower Closure Operator definition of collecting semantics of arithmetic expressions follows monotonicity, reductivity and idempotency.
	}

	\item{\textbf{Structural definition of the backward/bottom-up collecting semantics of arithmetic expressions} : revised definition and relational mapping that represent the changes in the environment variables it depends on.
	}

	\item{\textbf{Generic backward/bottom-up non-relational abstract semantics of arithmetic expressions} : it deals with getting the over approximations of backward collecting semantics.
	}

	\item{\textbf{Structural definition of the genericbackward/bottom-up non-relational abstract semantics of arithmetic expressions.}
	}

	\item{\textbf{Calculational design of the generic backward/bottom-up non-relational abstract semantics of arithmetic expressions} : uses compositional rules and compositions of various abstraction and concretization functions to realize the semantics. 
	}

	\item{\textbf{Ocaml Implementation of the primitive backward/bottom-up non-relational abstract arithmetic operations for initialization and simple sign analysis}.
	}

	\item{\textbf{Improving the non-relational analysis of boolean expressions using the backward analysis of its arithmetic subexpressions} : 

	The abstract interpretation of boolean expressions can be revised using the backward abstract interpretation of arithmetic expressions and can be parametried using comparison operators.
	}

	\item{\textbf{Calculational design of the revisited non-relational abstract interpretation of boolean expressions} : uses rule based derivation to find the abstraction of boolean expression.
	}

	\item{\textbf{Abstract arithmetic comparison operations forthe initialization and simple sign analysis} : deals with Generic abstract boolean equality.

	The calculational design ofthe abstract equality operation does not depend upon the specific choice of Lattice 'L'.

	the table with all the different possible combinations of data type and sign initialization is specified.
	}

	\item{\textbf{Local decreasing iterations} : the lower closure operaotor is better abstract interpretation for the  monotonous reductive type mapping on a given lattice.
	}

	\item{\textbf{The forward/top-down nonrelational abstractsemantics of arithmetic expressions is monotone} : 
	}

	\item{\textbf{The forward/top-down nonrelational abstractsemantics of boolean expressions is monotoneand reductive} :
	}

	\item{\textbf{reductive expression for boolean formulae} :
	}

\end{itemize}