\chapter{Operational Semantics}

Basic Lexer-Parser Theory has been explained with example.

\begin{itemize}
	\item{\textbf{Parsing and Translation} : Involves using a lexer to tokenize the given code. Parser to combine the tokens using the provided grammar rules/abstract syntax in a bottom-up or left-right approach.
	}

	\item{\textbf{Grammar semantics} : These convert the tokens into semantic values using what is characterized as semantic action.
	}

	\item{\textbf{Grammar rules \& Priority decidance} : involves assigning priority to operators.
	}

	\item{\textbf{Concrete syntax *} : Basically writing all possible variations differently.
	}

	\item{\textbf{Abstract Syntax *} : Writing similar type of grammar declarations at same place. may be as an Inducive definition.
	}

	\item{\textbf{Symbol Table} : list of all the program variables
	}

	\item{\textbf{Program Labelling} : labelling the different part of a program.
	}

	\item{\textbf{Operational Semantics of SIL} : Involves Small-step and Big-step operational semantics where we keep a record of every step variable's values and only final and initial values respectively.
	}

\end{itemize}
