\chapter{Program Specifications}

\begin{itemize}
	\item{\textbf{Program Semantics and Specifications} : semantics refers to the properties of program it does have, specifications refers to the properties the a program should have. program semantics include different type of Operational semantics, natural and reachability semantics.
	}

	\item{\textbf{Small Step Semantics} : It is a transition system of program states defined with transition rules and final states.
	}
	
	\item{\textbf{Trace Semantics } : records the sequence of program states.
	}
	
	\item{\textbf{Partial Trace Semantics } : records the sequence of program states given a precondition or a prefix trace.
	}
	
	\item{\textbf{Maximal Trace Semantics } : it is performing trace semantics for the state 'S' as well as it's negation 'not(S)'.
	}
	
	\item{\textbf{Big-Step Operational Semantics } : It is the semantics of transition system. It represents deciding rules for the Small-Step semantics.
	}
	
	\item{\textbf{ Natural Denotational Semantics} : decides the semantics of transition system for some blocked states. the 'natural' term possibly means the real case scenario where there are very possible blocked cases/states. 
	}
	
	\item{\textbf{ Natural Operational Semantics} : deciding Op-semanctics with care taken the non-termination is never possible.
	}
	
	\item{\textbf{Forward Reachability Semantics } : semantics of forward transition of program states.
	}
	
	\item{\textbf{Backward Reachability Semantics} : semantics of Backward transition of program states.
	}
	

	\textbf{From Abstract Interpretatin point of view, the different semantics are simply abstractions of each other.}


	\item{\textbf{Specification of program Semantics} : decideds the properties that a program semantics should have. Covers the Idea of Correctness of the program Semantics.
	}
	
	\item{\textbf{Trace Properties} : trace semantics should be a subset of correctness specification properties. 

	includes the idea of relational properties, partial correctness and total correctness.
	}
	
	\item{\textbf{State Properties} : It is also a weaker form of correctness of program specification properties. includes the idea of forward reachability, specification and runtime errors.

	example cases of proving state properties includes ideas like : there should exist no run-time error. 
	}
	
	\item{\textbf{Program Logics} : Formal Descriptions of program logic include representing programs specification in terms of sets, relation between sets and developing the trace semantics.
	}
	
	\item{\textbf{Set of States Predicate logic} : exlpined earlier, this deals with expressing the state transition in terms of disjoint mathematical sets.

	Developing Sets, we can understand the program specification in more better way.

	This can be further extended to also account for errors (on safe side we should definitely account for these error !!).

	It also involves extending the assignment properties too. 
	}
	
	\item{\textbf{State Relation Predicate Logic} : It involves extending the different state properties'. this means extending syntax rules, Assignment rules, predicate rules and trace rules.

	This extends the set-wise relation of semantical attributes of the program.
	}
	
	\item{\textbf{Trace Predicate Logic} : For  better assesment of trace semantics, we have used some of the predicates such as having a finite time interval, using atomic formulae.
	}
	
	\item{\textbf{Extending logic semantics} : This is performed by extending the syntax of the predicates, extending the assignment semantics of 'Control Variables'.
	}
	
	\item{\textbf{Linear Time Temporal Logic} : temporal logic specify out the execution trees, A linear time temporal logic essentially analyzes the execution trees by deciding traces one at a time.
	}
	
	\item{\textbf{Linear Time Temporal Logic syntax} : is similar to 'terms', includes definition for universal quantification, generalizations and negations. 
	}
	
	\item{\textbf{LTL expression system} : Includes LTL auxiliary operators - Eventually , Henceforth, and waiting .(all these words to be understood with respect to the time dependence nature of the program.)

	includes expressing temporal tautologies. (differnt expression forms)
	}
	
	\item{\textbf{Synchronous Languages} can be used to specify sets of finite traces.}

	\item{\textbf{comparative example of specification} : 
	a simple filter program is explained where by making it clear the temporal logic makes it difficult to understand each states. 

	it is thus difficult to auto-generate useful code.
	}
	
\end{itemize}
