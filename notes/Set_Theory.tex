\chapter{Set Theory}

Basics of Set theory is explained in this including its representations, terminology and definitions of Relations like transitive, reflexive, symmmetric-antisymmetric and relation equivalence .

\begin{itemize}
	
	\item{\textbf{Partitions} : Sub parts of set the complete it on conjuction.
	}

	\item{\textbf{Posets} : Partially Ordered Sets (Posets) have an ordering condition associated with each of its elements.
	}

	\item{\textbf{Hasse Diagram} : Diagram to represent the Posets in its trnsitive reduction form. Drawn in a bottom up faishon.
	}

	\item{\textbf{Encoding N with sets} is representing natural numbers using set notation. Expressed as : n+1 = \{ 0,1,2...n\}. 
	}
	
	Various Other things including function, pre-image concept and dual image  are explained too.

	\item{\textbf{Wosets} : posets with no infinitely strictly decreasing sequence (for natural numbers).
	}

	\item{\textbf{Equipotence} : It is an equivalence relation between 2 finite sets.
	}

	\item{\textbf{Cardinals} : That property of A which is inherent in any set B equivalent to A. Here two sets are called equivalent (or equipotent or of the same cardinality) if it is possible to construct a bijection (one-to-one correspondence) between them. 

	Mathematically it is possible to carry out mathematical operations on the cardinal sets like Adding, multiplying etc.

	}

	\item{\textbf{Ordinals} : This means the the following well-ordered set (Woset) has previous elements as its members.

	Ordinality means the Woset has 'is and element of' as the required ordering condition.

	Ordinal number is can be represented in set notation in a form expresses above.
	}

	\item{\textbf{Well Founded Sets}}

	\item{\textbf{Burali-Forti Paradox} : 
	The class 'O' of ordinals is not a set.
	}
\end{itemize}
